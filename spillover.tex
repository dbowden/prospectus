% !TEX root = 3_16.Rnw

\section{Outside Relationships and Dissident Network Dynamics}

While the knowledge that there is a general relationship between dissident network structure and the number of rebel groups that emerge is useful, it also begs the question of what shapes the dissident network. Furthermore, what explains changes in the dissident network over time? I expect that relationships between dissidents and outside actors, and paArticularly support from outside states, play an important role in shaping relations between dissidents. In the remainder of this section I discuss three mechanisms by which outside relationships might affect the dissident network.

\subsection{Signaling}

While virtually all dissident coalitions are likely to contain some underlying ideological divisions, the contours of these differences are not always obvious at the start of a conflict. Dissidents organize initially on the basis of a shared desire to replace the government, and perhaps a few central grievances. Some organizations declare a  comprehensive ideological platform from the start, including many communist and Islamist movements. Many, however, offer few clues to their preferences beyond removing the government or securing regional autonomy. For example, the Revolutionary United Front in Sierra Leone articulated few ideological positions or policy preferences, despite gaining control of the country's government for roughly a year. In some cases, rebel elites may even deliberately obfuscate their ideological leanings in hopes of co-opting as much of the dissident network as possible.

Even when some information about their preferences is available, the rebel coalitions that form initially may benefit from a ``honeymoon effect.'' Such dynamics have been shown to imbue various social entities with a temporary boost in favorable views. For example, U.S. presidents tend to experience their highest approval ratings in their first few months in office \citep{Erikson2002}. Even some citizens who voted against the president seem to overlook their reasons for doing so and temporarily offer their approval. A similar logic suggests that newly democratized regimes may be better able to withstand economic crises than their more established counterparts \citep{Bernhard2003}. We might then expect that opinion toward rebel groups exhibits similar patterns. In this line of thinking, new rebel groups receive considerable goodwill, perhaps as a reward for standing up to the government, or by offering a new and exciting direction for the country. Dissidents may overlook some of their disagreements with the rebel elite, projecting what they wish to see on the group. In the scenarios discussed thus far, rebels should enjoy strong support until or unless their true preferences begin to be revealed, clarifying to those with differing preferences that they are not well-represented by the group.

One process that may reduce uncertainty about dissident preferences is the procurement of support from outside states. External support tends to come from either a major power such as the US or USSR/Russia, or a neighboring state. While states often have ambiguous preferences regarding foreign civil wars, their broad alignments and values tend to be easily observable. Politically active individuals are likely to have opinions about such states, which will be ascribed to the non-state actors supported by said states. Actors that hold negative views of the supporting state are likely to adopt a less favorable view of their client states. Rebel groups may also be required to commit to certain positions or tactics in order to attract, or as a condition of, receiving support. For instance, a Syrian rebel group would be much more likely to receive support from the U.S. if it were to disavow jihadism.

In other situations the divisions among rebel groups may be well known prior to any outside involvement. This is especially likely among groups that have been previously active, or are comprised of pre-existing non-violent groups. Yet even in this scenario a rebel group may be able to achieve significant unity. This could be done through an institutional compromise, such as granting leadership positions to members of multiple factions. Alternatively, dissidents may simply agree to set aside their own differences until they have defeated the government. The addition of an outside state's interests may disrupt such equilibria. For example, the Karen National Union members share the goal of self-determination for the Karen people of Burma, but fragmented during the Cold War when some factions wished to pursue support from the US while others sought help from China.

In short, support from outside states should sow divisions among dissidents. In some cases outside support acts as a heuristic for dissidents assessing a rebel group's ideology and goals, making latent divisions more salient. In other cases outside intervention may deliberately manipulate a rebel group's alignments, creating tension. External support should thus promote the sorting of dissident networks into more ideologically homogeneous clusters. This places existing dissident coalitions at an increased risk of fragmentation, and can lead previously non-violent groups to enter the conflict, under a similar logic to fragmentation. Many factions of the dissident network may remain non-violent at the start of a conflict, assessing whether they can free ride off of existing groups. External support should tend to clarify the existing rebel group's positions, and increase the likelihood that members of the dissident network with substantially different preferences will mobilize. The preceding logic suggests that the effect of support should be fairly rapid, with updates to beliefs about group ideology coming immediately following the receipt of external support. While mobilizing support for a splinter organization and especially for the mobilization of an entirely new group may take time, the desire to do so should come quickly. Thus, we should expect the effect of external support to exhibit clear temporal patterns, with increases in modularity (polarization) of the dissident network occurring shortly after the initiation of external support to one or more members.

\noindent \textit{H5: The modularity of the dissident network should tend to increase after existing groups receive support from outside actors.}

\subsubsection{Resource Distribution}

Signaling dissident preferences is the only mechanism that might link relations with outside states to the structure of the dissident network. The resources that derive from external support may also shape relations between dissidents. External support is only likely to be given when the patron has its own agenda in a conflict. Providing support gives the patron leverage over its clients, and support may be contingent on assisting the patron in achieving its goals. For instance, the US has provided external support to numerous rebel and dissident organizations, at various times aimed at combating communism, the drug trade, or terrorism. A neighboring country to a civil war may offer support to combatants in hopes of supporting co-ethnics at the expense of other groups. In such cases, relations between dissident groups may again deteriorate when one or more groups receives external support, but in a process driven by the recipient group as a condition of receiving support, rather than in response to updated assessments of group goals.

% For instance, if a dissident group were to receive substantial material or monetary resources from an outside state, this could lead other dissident factions to pursue \textit{closer} ties with the recipient group, even leading them to set aside ideological differences. The disbursement of external support is often fairly centralized --- the external patron is likely to provide weapons to rebel leaders at a small number of locations, or transfer funds to a small number of accounts. Thus, to receive a share in these benefits, a non-recipient faction must ingratiate the recipient faction. While it may be possible for such groups to eventually establish independent ties with the outside supporter, demonstrating closeness to existing recipients is likely to be an effective strategy for demonstrating that such a relationship would be worthwhile to the supporter.

External support may also change rebel behavior in more general ways. \citet{Salehyan2014a} find that rebel groups with external support are significantly more likely to target civilians than those without. They suggest that as rebels become less dependent on the local population for material support, they become less constrained in dealing with them. Similarly, \citet{Weinstein2007} finds that rebel groups with large natural resource endowments are likely brutalize civilians, while groups without are likely to collaborate with them. While patron states who care about human rights, such as democracies, have some ability to reign in their rebel agents, in the aggregate external support increases rebel violence against civilians as even well-meaning supporters have difficulty enforcing their directives due to principal-agent problems \citep{Salehyan2014a}.

This process creates a pathway through which external support can provoke division within a rebel movement that is distinct from the ideological mechanism discussed above. Resources from external patrons, civilians, and the natural environment are somewhat interchangeable from the standpoint of rebel leaders, as each provides the materials or revenue needed to fight a war. From the standpoint of rank-and-file group members, however, these resource bases are not equally valuable. Rebel groups that rely on natural resources for revenue often provide members with a share of the profits \citep{Weinstein2007}. While the monetary value of civilian support may be comparatively lower, rebel soldiers are also likely share in many of the benefits such as food and shelter. External resources, by contrast, tend to be comprised largely of weapons, war materiel, and logistical support. Such resources may enhance a group's fighting capability, but provide fewer private benefits to individual members than natural resources or civilian support. It is likely that the receipt and distribution of external resources is more centralized compared to other resource bases. Thus, any private benefits that can be accrued from external support are likely to go disproportionately to rebel elites, and provide limited incentive for rebels to remain in the group.

While external support may bring few benefits to individual members, its behavioral consequences can provoke their opposition. \citet{Weinstein2007} suggests that natural resources lead groups to victimize civilians from an early time in their history. By contrast, external support is not always present early in a group's existence, and often begins suddenly at a later date. Thus, many groups foster collaborative relationships with civilians prior to receiving external support. The decision to turn away from civilian collaboration as external resources flow in may not be unanimous for rebel groups. Rebel leaders or their external sponsors may order increased violence against civilians as a means of increasing bargaining leverage \citep{Downes2006}. For rebel soldiers, such an order may require them to turn on civilians with whom they had previously collaborated, and in some cases their family, neighbor, or co-ethnics. In such cases, a substantial portion of a rebel group's membership is likely to oppose the changes that accompany external support.

Civilian victimization can also increase the probability that new groups will join the conflict. As the level of violence against civilians increases, incentives for individual dissidents to free ride diminish \citep{Kalyvas2014}. As the level of safety associated remaining neutral in a conflict decreases, so does the relative risk of participating in rebellion. Yet, dissidents facing violence from existing rebel groups will likely prefer not to join their attackers. Instead, they should prefer to join different groups, or form new ones if none exist. Thus, rebel violence against civilians should lead to a general increase in the number of rebel groups, as it provides incentives for both the splintering of the group engaging in violence, and for the formation of new groups.

To disentangle the potential mechanisms linking external support to dissident network structure, I propose two hypotheses. First, the signaling mechanism implies that any relationship with outside actors, regardless of the level of material support provided, should act as a heuristic for evaluating a dissident group's ideology and goals. Thus, we should see little difference in the effects of verbal and material cooperation with outside actors.

% \noindent \textit{H6a: Verbal and material cooperation between dissident groups and outside actors should be equally likely to increase the modularity of the dissident network.}

\noindent \textit{H6: Material cooperation between dissident groups and outside actors should be more likely than verbal cooperation to increase the modularity of the dissident network.}

Additionally, it is easily possible to test the proposition that the link between material support and changes in the dissident network is civilian victimization, as data civilian fatalities resulting from civil conflict are available for the post-Cold War era.

\noindent \textit{H7: The modularity of the dissident network should increase with the level of rebel violence against civilians.}

\subsubsection{International Network Spillover}

The preceding hypotheses examine the effect of any external support on the dissident network. Yet, it is likely the case that the specific identify of the state or organization providing support conditions its effects. Furthermore, it is often the case that multiple outside states intervene in civil wars. The relationship between these outside actors is likely to shape the effects of support as well. To account for these dynamics, I extend the dissident network framework to include a parallel network of external actors. This network is made up of all actors from outside the state that interact with either a dissident organization or the government. It is important to account for actors that align with the government in the external network (whereas the government and pro-government actors are not included in the dissident network) as the presence or absence of such ties is likely to shed light on the motives of the actors supporting dissident organizations. When both the government and dissidents are receiving support, supporters may be acting on a proxy war logic, with goals that are aimed at each other as much as the particular conflict in which they are contending. Support that is more one-sided, by contrast, is more likely to have goals specific to the conflict. The structure of this external network is then defined by the level of conflict or cooperation between its members. One might find a highly polarized external network, with one cluster of states supporting the government and a rival cluster supporting the largest rebel faction. In other cases the network might be highly fragmented, with several states that lack close ties each supporting a different actor in the conflict. Occasionally, outside actors may form a tight network, intervening overwhelmingly on one side of the conflict.

The structure of the external actor network should influence the structure of the dissident network through a process of spillover. In social network theory, spillover is a process through which the structure or content (norms, tactics, events, etc.) of one network shape those of another through an overlap in membership, or ties between networks. It is essentially a process of contagion through social ties. For instance, the U.S. peace movement adopted many of the tactics of the feminist movement in the 1980's, due at least in part to the considerable number of individuals who belonged to both \citep{Meyer1994}. \citet{Papachristos2015} show that individuals who share close social ties with gang members face an increased risk of being gunshot victims, regardless of whether they themselves are gang members. At the level of the international system, \cite{Maoz2011} finds that the network of IGO memberships exhibits spillover effects on the network of alliance ties, with the latter tending to follow the structure of the former.

I expect a similar dynamic to occur between the external actor and dissident networks. While the two networks will have little or no overlap in membership,\footnote{The only case in which they might is if a transnational organization such as al-Qaeda has a local presence in the dissident network, but also a broader international presence.} there are often significant connections between dissidents and outside actors, creating opportunities for the external network to influence the dissident network.\footnote{The reverse is also possible, but beyond the scope of this project.} As noted above, outside actors often provide material support to dissident groups. The recipient organization may be required to adopt the priorities of the donor, either as an explicit condition of receiving support, or by their own accord in hopes of attracting maximum support. As a result, the donor essentially leads the recipient groups to become a parallel node in the other network. Other outside actors may then decide to become involved in the conflict themselves, with their decision of whether to act and what group to support being contingent on their relationship with the initial donor. An outside state with close ties to the initial supporter may choose to free ride, or to contribute further resources to the same recipient group. An outside state with weak or negative ties to the initial supporter, however, is likely to support a different dissident group. As the recipient organization aligns with the donor, the dissident network begins to further mirror the polarized structure of the outside actor network. In conflicts where dissidents receive considerable external support, the dissident network should closely resemble the network of outside supporters.

\noindent \textit{H8a: The lagged density of the external support network should have a positive relationship with the density of the dissident network.}

\noindent \textit{H8b: The lagged modularity of the external support network should have a positive relationship with the modularity of the dissident network.}

\begin{table}
\centering
\begin{tabular}{llll}
\hline
 & DV & IV & Direction\\
\hline
H1 & \# rebel groups & dissident network density & $\downarrow$ \\
H2 & \# rebel groups & dissident network density X modularity & $\uparrow$\\
H3 & \# rebel groups & calls for resolution & $\uparrow$\\
H4 & \# rebel groups & initial network density & $\downarrow$\\
H5 & dissident network modularity & external support onset & $\uparrow$\\
H6 & dissident network modularity & external material support & $>$ verbal support\\
H7 & dissident network modularity & rebel violence against civilians & $\uparrow$\\
H8a & dissident network density & external network density & $\uparrow$\\
H8b & dissident network modularity & external network modularity & $\uparrow$\\
\hline
\end{tabular}
\caption{Summary of Hypotheses}
\end{table}
