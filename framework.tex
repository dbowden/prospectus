\section{Theoretical Framework: The Dissident Network}

I conceptualize dissidents during periods of civil unrest as comprising a network \citep[for similar approaches, see][]{Zwerman2005,Metternich2013}. The nodes of this network are comprised of various organizations, and might include rebel groups, political parties, labor unions, student organizations, ethnic militias, opposition media outlets, and shadow institutions. While I am primarily interested in groups with some willingness to use violence, nonviolent groups also require consideration as they may play important roles in shaping opportunities for violent groups, and their existence may signal the presence of a latent violent group. For instance, an opposition newspaper is highly unlikely to engage in violence itself, but if it expresses negative views of existing rebel groups, this might indicate broader demand for a new rebel group representing different interests. For illustrative purposes, the member organizations of the Palestinian dissident movement are described in Table \ref{paldiss}.\footnote{The network is defined using the criteria described in the research design section below.} One can see that there are a number of political parties and militant groups with diverse ideologies. Several governing institutions associated with the Palestinian Authority are also included as they exhibit hostility toward the Israeli government. Similar governing institutions are common among separatist movements, and rare among other dissident movements. While they are unlikely to engage in violence themselves, their inclusion in the network is important as they may play important roles in mediating the relationships between other actors.

The cohesion and integration of the dissident network can vary greatly both across countries and over time. At one extreme, some secessionist movements are represented by a single group exercising tight control over both political and military affairs. At the other, dissidents living under a repressive regime might be unable to organize and be little more than a collection of like-minded individuals. Most cases lie somewhere between, with dissidents split into several organizations that may at times engage in significant cooperation, but remain formally separate. In the Palestinian case, for example, Hamas and Fatah co-existed as members of the Palestinian Authority for a time, but became sharply divided following the death of Yasser Arafat in 2004.

\begin{table}
\centering
\begin{tabular}{ll}
Group Name & Description\\
\hline
Palestinian Liberation Organization & Originally a violent insurgent group,\\
 & now recognized as representative of Palestinian\\
 & people in negotiations, IGOs\\
&\\
Fatah & Secular, relatively moderate political party\\
&\\
Tanzim & Militant wing of Fatah\\
&\\
al-Aqsa Martyrs' Brigades & Militant group with close ties to Fatah\\
&\\
Islamic Jihad & Political wing of radical Islamist\\
& insurgent/terrorist group\\
&\\
Al-Quds Brigades & Militant wing of Islamic Jihad\\
&\\
Hamas & Hardliner party that opposes two-state solution,\\
& moderate emphasis on Islam\\
&\\
Izz ad-Din al-Qassam Brigades & Militant wing of Hamas\\
&\\
Popular Front for the Liberation of Palestine & Leninist organization that opposes two-state\\
& solution\\
&\\
Democratic Front for the Liberation of Palestine & More moderate Leninst organization\\
&\\
Palestinian Legislative Council & Legislature of Palestinian Authority, 1996--2007\\
&\\
Palestinian Preventive Security Agency & Security force of Palestinian Authority\\
&\\
WAFA & News agency of Palestinian Authority\\
&\\
Army of Islam & Radical Islamic group aimed at establishing\\
& caliphate in Palestinian territories\\
\hline
\end{tabular}
\caption{The Palestinian Dissident Network, 1995--2014}
\label{paldiss}
\end{table}

In any case, the dissident network has a common interest in obtaining political change, which generally creates an incentive for cooperation. Only in rare cases will a single element of the dissident network be strong enough to defeat the government on its own. At the opposite extreme are cases in which even the totality of dissident capabilities are not enough to defeat the government. Often, however, cooperation between different elements of the dissident network should lead to meaningful improvements in the odds of achieving political change. For instance, a pair of rebel groups may be weaker than the government forces individually, but in combination would have the advantage. Even if forming a coalition does not give dissidents enough strength to topple the government, it may enhance their leverage in a negotiated settlement. Despite this common interest, however, dissidents often fail to cooperate. In the remainder of this section I explore the ways in which dissident interests diverge, presenting a barrier to cooperation.

While dissidents share a common interest in removing the incumbent government, they do not necessarily agree on much else. Some dissidents may prefer that a new government emphasize or favor a certain ethnic or religious identity. This is especially common among groups pursuing secession or regional autonomy, which are nearly always based on particular identities. Yet, even within potential secessionist regions, some dissidents may prefer to reform the central government rather than secede from it. Bids to overthrow the central government can also emphasize one identity group at the expense of others. Dissidents who do not share this identity would oppose such an outcome. For example, Ansar al-Islam sought to restore Sunni dominance in Iraq and was opposed by Shi'a militias such as the Mahdi Army. The dimensions of identity on which dissidents mobilize is also dynamic in some cases. The Arakan separatists in Myanmar share an ethnic identity, but are divided into factions along Muslim-Buddhist lines. Dissidents can also be divided by geography, class, or sector. Rural dissidents might make land reform their top priority in a post-war government, whereas urban dissidents might care more about welfare or modernization programs. Some dissidents hope to take control of a government with strong centralized authority, while others hope to procure greater regional autonomy as a consequence of the war. Broader left-right ideological divisions may also be present. Even when dissidents largely agree on goals, there are likely to be divisions between hardliners and moderates, who will be more willing to accept compromises and less willing to adopt extreme tactics. Finally, even dissidents who largely agree on questions of policy will still find themselves in competition over the power and private benefits of government \citep{Christia2012}, which are subject to rival consumption.

\begin{table}
\begin{tabular}{lll}
Dimension & Conflict & Example\\
\hline
Ideology & Vision for new government & Free Syrian Army vs. Al-Nusra Front (Syria):\\
& & secular vs. Jihadist\\
Identity & Religious / ethnic divides & Arakan separatists (Burma):\\
& & Muslim vs. Buddhist\\
Bargaining Orientation & Hardliners vs. moderates & Hamas vs. Fatah (Palestine):\\
& & acceptance of two state solution vs. not
\end{tabular}
\caption{Potential Divisions in the Dissident Network}
\end{table}

\subsection{Dissident Preferences}

Dissidents face a choice between pursuing their interests through peaceful strategies, or resorting to violence. I argue that this calculation is shaped by two underlying preferences, which then interact with features of the dissident network and the broader conflict environment. First, dissidents should prefer an outcome that is as close as possible to their ideal vision of a new government. While a dissident faction might find some benefit if a group with very different priorities succeeding in overthrowing the government, \textit{ceteris paribus} such an outcome is likely to be inferior to overthrowing the government themselves and prioritizing their own interests. For example, a communist insurgency would experience only modest benefit if other rebels established a new democratic regime in which communists are unlikely to attract significant electoral support. A group hoping to replace a monarchical regime in the Arabian Peninsula with an Islamist one would similarly find little satisfaction if another group succeeded in establishing a military regime. Thus there is a general incentive for dissident factions to mobilize separately any time their interests are not well-represented by existing violent groups.

This incentive is undercut, however, by the second preference. Dissidents should prefer to minimize the costs of advocating for their position. Violent mobilization brings the obvious risk of death or injury on the battlefield, as well as the possibility of harsh punishment if the rebellion fails. In some cases joining a rebellion is also subject to high opportunity costs, if participation in the normal economy remains a possibility or a rebellion prevents one from defend their family. Thus, the free rider problem that has been widely discussed at the individual level in the context of civil war \citep[e.g.][]{Lichbach1995,Wood2003} is likely to apply at the group level as well. Dissident factions must weigh the value of pursuing their specific set of interests against the costs of mobilization. The value of separate mobilization is likely to be positive only when a group's interests are substantially different from those of groups that have already mobilized.

\subsection{Rebel Movements}

Ultimately, a subset of the dissident network may elect to resort to violence. The groups that do so form a rebel movement in parallel to the dissident network. Like the dissident network, rebel movements can vary greatly in structure. In some cases only one dissident organization resorts to violence, resulting in a single, relatively cohesive group. In other cases a single rebel group may form, but encompass several dissident factions. Such groups are likely to have internal divisions and face relatively high risk of fragmentation. It is also possible for multiple groups of either type to emerge. Rebel movement structure is also likely to vary over time within specific conflicts. Few conflicts have multiple rebel groups in their first year, yet more than 40\% have multiple rebel groups at some point. The structure of rebel movements can be altered through several processes, which I outline in the remainder of this section.

\subsubsection{Fragmentation}

Ideological divides can occur both between separate groups and within a single group. In the former case, a dissident network might include both a radical left student group and a center left labor union. In the latter, a self-determination movement might include members who will accept nothing short of independence, as well as members willing to settle for increased regional autonomy. Ideological divisions between already-separate groups create the possibility that each will engage in violent mobilize separately, rather than forming a single rebel group. Divisions within a single group create the possibility that the group will fragment into multiple successor organizations.
% I enumerate the differences between these processes in the remainder of this section.

Dissident group members who disagree with their leadership's position on important issues face a difficult choice. Forming a new organization from components of an existing one avoids many of the challenges of building a rebel group from scratch. Whereas new groups must convince recruits to take extraordinary risk in joining the conflict \citep{Lichbach1995}, splinter organizations can recruit from a pool of individuals already participating in the war. Relatedly, rather than having to draw recruits from disparate parts of society, a potential splinter organization may have a well-defined support base within the parent organization. For instance, if a rebel group fractures along religious lines, the recruiting pool for the splinter organization is quite clear. Finally, a splinter organization may be able to retain control of some of the parent group's resources. If the group procures resources through collaboration with civilians, the individual members involved may be able to continue their relationships with the new organization. Similarly, if a faction of a rebel group controls a segment of the resource extraction process, they might continue to do so under a new organization.

Yet, fragmentation also carries significant risk. The parent organization may engage in reprisals against the splinter organization, hoping to deter further fragmentation. Even if the resulting organizations do not fight, each is weaker and thus less likely to succeed than they were before. In some cases control of a group's resource base is highly centralized, and unlikely to be co-opted by a splinter organization. This might be the case for groups involved in sophisticated resource extraction, such as oil extraction or drug smuggling. Given these dangers, members of existing groups should be risk-averse. In most cases a significant ideological difference or identity-based schism will is required to motivate fragmenting. This effect is contingent on the portability of the existing group's resource base, however. When members can expect to enjoy the same level of resources in a new group, the threshold needed to motivate fragmentation is much lower.

\subsubsection{Expansion}

Fragmentation is not the only path to multiple rebel groups. When there are divisions between an existing rebel group and other elements of the dissident network, new groups may enter the conflict. Differing political goals create an incentive for dissident factions to mobilize separately \citep{Wolford}, expanding the conflict. The value of doing so, however, is diminished by several factors. First, as the number of factions with whom a group would have to share power increases, the value of separate mobilization becomes less likely to outweigh the costs of fighting \citep{Christia2012,Wolford}. Second, as the probability of being in the winning coalition decreases, the value of mobilizing separately rather than participating in a coalition decreases \citep{Christia2012}. Finally, as the magnitude of ideological differences decreases, the value of mobilizing separately rather than free riding off of other groups diminishes \citep{Lichbach1995,Metternich2013}. Still, there will be some situations where the preferences of one faction are sufficiently different from existing rebel groups that they have incentive to enter the conflict independently.

Like fragmentation, entering an ongoing conflict is in many ways easier than initiating a conflict during a condition of peace. First, joining a conflict generally occurs in a sharply differing environment than conflict initiation. Initial decisions to launch a rebellion tend to occur amidst some amount of unrest, such as protests or riots, economic turmoil, or conflict in neighboring states. Nevertheless, the decision to initiate a war by definition does not occur during times of outright war, whereas joining decisions generally do.\footnote{This distinction may not hold in the case that civil war violence is highly localized, leaving some areas mostly unaffected by the fighting.} The barriers to entry for a new rebel group are likely to be lower during an ongoing conflict than during peace. While poor economic performance is associated with conflict onset \citep{Collier2004}, recruiting rebels to initiate a war requires individuals to exit an economy that is at least somewhat functional. During times of war, however, economic activity reduces dramatically \citep{collier99}. Thus, the opportunity cost of joining a rebellion is likely to be much lower when a war has already begun. Similarly, while the risk of personal harm makes participation in rebellion more costly than non-participation before a war has begun, once fighting has initiated rebellion becomes relatively less costly as civilians face a high probability of experiencing violence. In fact, in some cases joining a rebel group may be safer than remaining neutral \citep{Kalyvas2014}. Finally, whereas the initial formation of a rebellion brings high risks of repression and arrest, once a civil war has begun the government is likely to be preoccupied with the existing group, creating opportunities for new groups to form with relatively little resistance.

% In addition to dramatically altering the material costs of participation in rebellion relative to maintaining civilian status, the onset of fighting reveals information relevant to mobilization decisions. In peacetime, dissidents may be hesitant to launch rebellions due to uncertainty about their likelihood of success. The government's strength and tactical choices, as well as the number of individuals in society willing to support a rebellion, may be difficult to judge. Fighting between the government and the initiating rebel group is likely to reveal information on these dimensions. The government's strength, the level of brutality it is willing to adopt, and the extent of societal opposition to the government are likely to be clarified by the initial fighting. While the information needed to predict conflict outcomes will never be complete, conflict joiners should generally have less uncertainty about their likelihood of success than conflict initiators.

In short, the costs of forming a new rebel group during a conflict should be less than the costs of splintering from an existing one. Whereas splintering requires dissidents to leave a group that has already organized and may be supplying them with some form of private benefits with the lurking threat of reprisal, the decision to join a conflict is weighed against the value of being a civilian in wartime. Thus in many cases, relatively small ideological differences should be enough to motivate a group to enter an ongoing conflict, and the phenomenon should be more common than fragmentation.

% Do I need something on risk?

\subsubsection{Consolidation}

Finally, in some cases rebel movements become less, rather than more complex. There are numerous examples of previously independent rebel groups merging to form a new umbrella organization. For instance, the United Islamic Front for the Salvation of Afghanistan, commonly known as the Northern Alliance, was an amalgamation of several groups including Jam'iyyat-i Islami-yi Afghanistan and Eastern Shura created in 1996 to oppose the Taliban. Consolidation generally occurs after rebel groups have spent some amount of time operating independently. This suggests that the groups either saw value in mobilizing separately, perhaps due to differing interests, or they originated in different geographic areas. The decision to consolidate is thus likely endogenous to conflict dynamics. If over the course of the war dissident interests evolve or are revealed to be similar to one another, they may elect to form an alliance out of a burden-sharing logic \citep{Sandler1980}. In other cases, however, alliance formation may reflect a capability aggregation logic. In the Northen Alliance example, member groups merged in response to the strength of the Taliban, which made the prospects of achieving success while mobilizing individually appear grim.
