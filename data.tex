% !TEX root = 3_16.Rnw

\section{Research Design}

\subsection{Primary Data Source}

Social network data is constructed from the Integrated Crisis Early Warning System (ICEWS) \citep{Boschee2015,OBrien2010}. ICEWS is a DARPA-funded initiative aimed at developing predictive models of both international and internal conflicts. The publicly available component of the project is an event data set covering the entire world for the period 1995--2014, with new data being released on a 12 month delay. The ICEWS data is machine-coded, using natural language processing\footnote{JABARI, a commercial variant of Phil Schrodt's Text Analysis By Augmented Replacement Instructions (TABARI) program \citep{Schrodt2011} is used to parse the articles.} to identify politically relevant events described in news articles. ICEWS draws on 281 print media sources, including large news agencies such as \textit{Agence France-Presse}, \textit{BBC}, \textit{Reuters}, and \textit{The New York Times}, local newspapers from a variety of countries including \textit{Al Raya} (Qatar), \textit{El Cronista} (Argentina), and \textit{The Egyptian Gazette}, and a number of national news agencies. When necessary, articles are translated to English before being coded. The software parses the first half of articles to look for events, which are coded using the Conflict and Mediation Event Observation (CAMEO) coding scheme \citep{Schrodt2007}. The coding scheme captures several pieces of information in the form of ``who did what to whom.'' CAMEO offers a major advantage over alternative datasets in that it includes a wide range of cooperative and conflictual behaviors, whereas most event data includes only conflict. Furthermore, it captures the direction of the event, whereas most event data does not distinguish between the initiator and target of an action. These attributes make it ideally suited to identifying network ties. Finally, CAMEO events are translated to the Goldstein Scale \citep{Goldstein1992}, which places events on a continuous scale ranging from material conflict (-10 to -5), to verbal conflict (-5 to 0), to verbal cooperation (0 to 5), to material cooperation (5 to 10), allowing for easy summarization of relationships between actors.

ICEWS includes roughly 13.5 million events over the period 1995--2014, spanning 281 countries and territories. Importantly for this study, ICEWS is perhaps the only dataset that captures a significant number of interactions \textit{between} rebel groups and other dissidents. Still, the dataset has several flaws and limitations. First, it covers a limited temporal domain. In fairness, this problem is common to most of the major events datasets, and ICEWS boasts a larger spatial domain than most of its competitors.\footnote{The ACLED data covers only Africa for the period 1997--2015, the SCAD data covers Africa and Central America for the period 1989--2014, and the UCDP GED data covers Africa, the Middle East and Asia for the period 1989--2014. Only the Cline Center's SPEED data offers significantly greater coverage, spanning the entire world for the period 1945--2014. Unfortunately, SPEED currently lacks sufficient detail to perform analyses disaggregated by actor.} This problem could be addressed in part by employing a dataset that use the same coding scheme and covers the Levant for the period 1979--2015 \citep{Schrodt2015}. Beyond this, however, the only options might be constructing networks from original data or building sparse networks from less granular conflict data such as the UCDP Dyad-Year data. Second, while the methodology of coding events from news articles is very refined, procedures for identifying duplicate events resulting from a story being covered by multiple media outlets are lacking. This problem can be mostly resolved, however, by removing observations that have identical source and target actors, dates, and CAMEO interaction codes. Third, the project was originally intended to predict conflict onset at the country level, and is not ideally suited to being disaggregated by subnational actor. Whereas some datasets match events to a fixed universe of actors with unique identifiers, ICEWS provides the most precise names discernible from the source article. Often, this results in a level of granularity beyond what most scholars would prefer. For instance, the source actor in a verbal cooperation event might be listed as ``George Habash'' rather than the ``Popular Front for the Liberation of Palestine.'' This sort of problem can be resolved by recoding individuals to their groups. More problematic are the underspecified cases in which actors receive generic names such as ``Rebel'' or ``Insurgent,'' usually resulting from news reports that do not specify a specific actor. In approximately 25\% of interactions between dissident groups, one or both actors are unidentified. It may be possible to develop a simple algorithm that identifies the group involved in unattributed events using the event date, spatial coordinates, target, and event type. Events that are classified with a high degree of certainty could then be recoded. For instance, an insurgent group operating in Northeastern Nigeria in 2014 could likely be classified as Boko Haram with a high degree of certainty. Alternatively, one could simply drop the unattributed cases and still be left with millions of observations. However, this will very likely introduce bias into the sample, with smaller groups being especially likely to be unidentified. Finally, ICEWS contains a few imbalances in sampling, with the few years having fewer events than subsequent years. However, beginning in the late 1990's the number of events per year is quite stable, with variation that is consistent with the level of conflict worldwide. ICEWS also oversamples countries that receive extensive coverage in English-language media. Any politically-relevant event in the Western World or East Asia is likely to be captured, while data in locales including most of Africa is comparatively sparse. It is crucial to note that this sparseness is relative to itself in other countries, however, as ICEWS captures more events than any other dataset. While these imbalances are problematic for analyses involving the volume events, they are less concerning for the present application, which summarizes the character of events rather than their frequency.

\subsection{Constructing the Dissident Network}

In theory, one approach to measuring dissident interests would be to hand code groups on the basis of manifestos and public statements. Yet, this would require identifying not only the interests of rebel groups, but also those of \textit{potential} rebel groups. Finding sufficient information on such groups across a variety of conflicts is unlikely. Furthermore, these groups are likely to be so great in number that coding data for each of them would be quite arduous. Networks have been used to measure interest similarity in a variety of contexts, including the US Congress \citep{Andris2015}.

To construct a dissident network from the ICEWS data, I draw heavily on the methodology of \citet{Metternich2013}, but also depart from their approach in several ways. The first step in this process is identifying the dissident groups in a given country. While ICEWS provides a ``sector'' code providing a generic categorization of actors (e.g. rebel, terrorist, etc.), these codings lack a clear conceptual scheme and several thousand sector values appear in the data. Thus, I follow \citet{Metternich2013} in defining dissidents behaviorally. I code any group as a dissident any time it has more conflictual than cooperative interactions with the government in a period (i.e. the mean Goldstein score of its interactions with the government is less than zero). The group then remains in the dissident network until either it experiences a period of at least 18 months in which it does not appear in any events, or it experiences a period in which its interactions with the government are more cooperative than conflictual. This is a somewhat more permissive definition that that of \citet{Metternich2013}, who remove groups from their anti-government networks after six months of no conflictual interactions with the government. Yet, that approach likely excludes such actors as rebel groups that retreat to remote areas, or that agree to ceasefires without demobilizing. These groups very likely continue to be opposed to the government, and remain active as organizations, and thus belong in the dissident network. This approach identifies virtually all rebel groups and militias as dissidents, as well as many opposition political parties and NGOs, and a few opposition media organizations.

Once the nodes in the dissident network have been identified, it is necessary to define the ties between them. Whereas \citet{Metternich2013} code binary network ties between dissident groups whenever they have cooperative events in a period. Yet, this approach ignores a great deal of variation among actors with no cooperative relationship. Many of these dyads are groups that simply never interact. Yet these cases are treated in an identical manner to dissident dyads that are in conflict with one another. To capture this difference, I employ a signed network, with network ties taking a positive value when the mean Goldstein score for a period is positive (signifying a relationship that is more cooperative than conflictual), and a negative tie when the mean Goldstein score for a period is negative (signifying a relationship that is more conflictual than cooperative). Second, even if the cleavages between dissidents are likely to evolve over time, it is unlikely that such relationships change on a monthly basis. Additionally, for many dissident dyads, interactions are few and far between. Yet, the existence of cooperation in one month and the absence of any interaction the next is unlikely to be indicative of a real change in the underlying relationship. Thus, rather than redefining network ties each month, I use the most recent monthly average of interactions for which there is data, only coding the end of a relationship after 12 months of no interaction.

\subsection{Constructing the External Support Network}

The external support network is constructed using similar rules to the dissident network. Any outside state or organization that has interacted with at least one member of the dissident network in the past 18 months is included. Both cooperative and conflictual relationships with dissidents are likely to contain useful information about a dissidents group's underlying preferences, and thus both are included. As with the dissident network, ties within the external network are defined by the interactions captured by ICEWS. Outside actors that have more cooperative interactions than conflictual ones receive a positive tie, while actors that have more coflictual than cooperative relations receive a negative tie.

\subsection{Case Selection}

ICEWS covers many countries and periods that do not experience significant levels of civil conflict. While it captures many types of information not found in other datasets, ICEWS does not include information about fatalities, meaning that outside data is needed to define conflict periods using traditional fatality-based measures. I thus use the UCDP/PRIO Armed Conflict Data \citep{Pettersson2015a} to define the sample. UCDP codes a civil conflict as occurring in any year in which fighting between a non-state actor and the government produces at least 25 fatalities in a calendar year. 68 countries experienced a conflict meeting this threshold during the period 1995--2014, and there are 536 total country-years with conflict. The year prior to conflicts is included as well to calculate network attributes.

\subsection{Dependent Variable}

\textbf{Number of Rebel Groups} The dependent variable in this study is, the number of rebel groups active in a conflict. The measure is constructed from the ICEWS data after the recodings of actor names discussed above has occurred. It consists of the number of non-state groups engaged in material conflict against the government in a given period (I plan to conduct analyses using both monthly and yearly data). Events falling into this category range from destruction of property to non-lethal assaults to suicide bombings to organized battles. This measure does not account for the emergence of new rebel groups; it simply counts the number of groups active against the government in a given year. It is easily possible to construct alternative specifications of this measure, for instance by including only lethal conflict events, or counting only groups that are involved in a certain number of conflict events. Setting the threshold at any material conflict with the government produces considerable variation. For instance, as many as nine Palestinian groups and as few as zero are engaged in violence against the Israeli government at various times, with the average falling between two and three.

\textit{Alternative Measures} One potential criticism of using ICEWS to measure the number of rebel groups in a conflict is that it is largely dependent on media characterizations of which organizations should be treated as independent, and which are merely factions of larger groups. The UCDP Dyadic Conflict Data is human coded, and defines groups separate groups where an organization itself uses a separate name from other actors. A measure of the number of rebel groups present in a given period constructed from the UCDP data should therefore provide enhanced confidence in the validity of the results.

Given that the theory proposed above expects greater numbers of rebel groups to mobilize against the government as dissident preferences become more diverse, it may be useful as an alternative DV or as an extension to assess the network ties between the groups that participate in violence in a given period. For instance if the broader dissident network shows a low density of ties, indicating diverse interests, the emergence of multiple rebel groups would only be consistent with the theory if said groups were not closely networked. Accounting for the network connections between violent groups, perhaps by dividing the number of groups by the level of connectivity (the proportion of actual ties relative to possible ties) would ``penalize'' the violent group count as the level of connectivity between them increases.

\textbf{Dissident Network Density} \textit{H8a} predicts that the density of the dissident network will tend to reflect that of the external support network. This concept is captured by the edge density of the network, which is the total number of actual ties divided by the total number of possible ties.

\textbf{Dissident Network Modularity} \textit{H8b} predicts that the modularity of the dissident network will tend to reflect that of the external support network. Measuring this concept first requires the application of a community detection algorithm to the network. I employ the walktrap algorithm, which begins by treating every network node as a separate community, then aggregates them based on a random walk over network ties. Modularity is the sum of the probability of a tie in the network actually falling in a cluster minus the probability of a tie falling into a cluster randomly. The modularity score under this method is the highest modularity value observed between the step in which every node is a community, and the entire network is a single community.

\subsection{Explanatory Variables}

\textbf{Dissident Network Density} \textit{H1} predicts that the degree of connectivity of the dissident network will shape the structure of violence against the government.

\textbf{Dissident Network Modularity} \textit{H2} predicts that modular networks, with two or more distinct ``communities'' of dissidents, should be especially likely to be produce multiple rebel groups.

\textbf{Calls for Conflict Resolution} \textit{H3} argues that the logic of a dissident group engaging in violence individually is driven in part by the value of participation in the peace process. It also suggests that calls for conflict resolution by actors within or outside the conflict may serve as a leading indicator that the peace process is likely to begin, providing an impetus for groups to mobilize. ICEWS captures many events in which an actor calls for a conflict to be resolved. The CAMEO coding scheme has separate codes for instances in which an actor calls on others to settle a dispute, to negotiate, and to accept mediation. Roughly 9,000 events of this type appear in the data.

\textbf{External Network Density} \textit{H8a} predicts that through processes of spillover, the density of the external actor network should influence the density of the dissident network, with the latter increasing its resemblance to the former. I calculate a measure of the edge density for the external network in an equivalent manner as for the dissident network.

\textbf{External Network Modularity} \textit{H8b} predicts that the modularity of the external network will spillover to the dissident network as well. Again the measure is calculated in an identical that of the dissident network.

\subsection{Control Variables}

\textbf{Dissident Network Size} While I expect the structure of the dissident network to shape mobilization patterns, it is also likely the case that larger networks produce more rebel groups. I therefore control for the number of nodes in the dissident network.

\textbf{Conflict Intensity} The number of active rebel groups is also likely tied to conflict dynamics. While it is unclear whether the number of rebel groups is a cause or consequence of conflict intensity, the two phenomenon appear to be related \citep{Akcinaroglu2012a}. First, I include a count of the total number of material conflict events between dissidents and the government captured in the ICEWS data. Second, as ICEWS does not include fatality counts, I include the total number fatalities resulting from fighting between the government and rebels, as measured by the UCDP Battle-Related Deaths Data \citep{Sundberg2008a}.

\textbf{Politically-Relevant Ethnic Groups} One factor that could lead dissidents to form multiple rebel groups instead of a single one is ethnic diversity. Numerous scholars have found that co-ethnics are able to cooperate effectively, while diverse groups often struggle to do so \citep[e.g.][]{Habyarimana2012}. Beyond that, dissidents may simply have interests that are unique to their ethnic group, incentivizing mobilization along ethnic lines. I include a measure of the number of ethnic groups experiencing discrimination from the Ethnic Power Relations Data \citep{Vogt2015}, as well as the total number of politically-relevant ethnic groups.

\textbf{Population} One might imagine that the number of rebel groups is at least in part a function of the size of the country. This would be especially likely if some factor placed a constraint on the size of rebel groups, as might be the case if rebels sought to maximize shares of resource revenues. Thus, I include a measure of country population from the World Development Indicators \citep{Bank2015}.

\textbf{Area} Similarly, countries with larger geographic areas might offer more opportunities for multiple rebel groups to operate. For instance, larger countries may have more natural resource deposits, or more rough terrain that provides a haven for insurgents \citep[see][]{fearonlaitin03}. The data again come from the World Development Indicators.

\textbf{Democracy} \citet{Wolford} find that democratic states tend to have more rebel groups than others. This might be the case as the interests of political groups are more easily identifiable in an open system, and individuals with similar interests are already organized into parties and interest groups. I include a binary indicator of whether the conflict-year occurs in a democratic state using data from the Polity IV project \citep{Marshall2012}.

Finally, the presence of natural resources may shape the structure of rebel movements. \citet{Weinstein2007} finds that the attributes of individual rebel groups are powerfully shaped by whether they formed in the presence of natural resources, with groups that have significant access to resources being more likely to abuse civilians. It may be that these sorts of groups are also more likely to invite challenges from other groups of dissidents. Alternatively, resources could create incentives for rebel groups to remain small, so as to maximize the share of resources distributed to each member. For these reasons, I control for the presence of a number of natural resources previously shown to be associated with conflict. Each of these measures is aggregated into a binary indicator of whether the resource is present in a country.

\textbf{Secondary Diamonds}: Diamonds are one of the most prominent forms of natural resources known to be used by rebel groups, including those in Sierra Leone and Angola. Secondary diamond deposits are those present on the earth's surface, and often found in river beds or rock formations. Whereas the extraction of primary diamonds requires sophisticated mining equipment, secondary diamonds are thought to be easily ``lootable'' by rebel groups, as their extraction requires only simple tools \citep{Lujala2005}. Geocoded data on secondary diamonds comes from the Diamond Dataset \citep{Gilmore2007,Lujala2005}. The data cover the entire world, and lists lootable diamond deposits in 40 countries, with 884 total deposits. Each of the resource variables is aggregated to a binary indicator of whether the resource is present in a country.

\textbf{Gemstones}: Precious gems other than diamonds are included in GEMDATA \citep{Lujala2008}. The data includes 1,022 sites where gemstones have been discovered, spanning 53 countries. The gems included are: ruby, sapphire, emerald, aquamarine, heliodor, moganite, goshenite, nephrite, jadeite, lapis lazuli, opal, tourmaline, periodit, topaz, pearl, garnet, zircon, spinel, amber, and quartz.

\textbf{Placer Gold}: Placer gold is considered to be easily lootable, as it is found primarily in riverbeds and is easily extractable. The data comes from GOLDATA \citep{Balestri2012}, and includes 550 locations with lootable gold in 48 countries.

\textbf{Onshore Oil Deposits}: Oil has also provided a source of financing for some rebel groups, including the Islamic State in Iraq and Syria. PETRODATA \citep{Lujala2007} lists 424 onshore geocoded oil deposits with known production in 85 countries.

\textbf{Drugs}: Numerous rebel groups procure funds by controlling the drug trade. DRUGDATA \citep{Buhaug2005} provides geocoded polygons of areas in which the cultivation of coca bush, opium poppy, and cannabis are known to occur.

\subsubsection{Models}

To test my hypotheses, I follow an approach common among studies of network effects \citep[e.g.][]{Maoz2011,Metternich2013,Papachristos2015} by using the network measures as covariates in standard regression models. Advantages of this approach over methods specific to network analysis are that it easily allows for the use of control variables, and it facilitates conventional hypothesis testing. Furthermore, I am ultimately interested in conflict-level outcomes separate from the dissident network, rather than group-level phenomena where the interdependence between units would require special consideration. The dependent variable for the first four hypotheses is the number of rebel groups active in a given period. As this is a count variable, poisson or negative binomial regression will be used. The dependent variables for the remaining six hypotheses are network attributes. Both network measures --- edge density and modularity --- are probabilities, and thus fractional probit regression is used.

For robustness, I also test the last six hypotheses using an exponential random graph model (ERGM). The ERGM approach recognizes that there is a certain amount of randomness in the formation of social networks. In this study for example, the presence of a cooperative tie between two groups in one month and the absence of one in the next is not necessarily indicative of a change in the underlying relationship. Instead, the two groups may not have interacted purely by chance. ERGM seeks to resolve this by treating any social network as one of many possible realizations. This allows the researcher to perform statistical inference on a variable, testing whether it produces a different network structure than would emerge by chance.

\subsection{Qualitative Case Studies}

To bolster the evidence produced by the two quantitative analyses described above, I will conduct qualitative case studies of multiple rebel movements. The quantitative studies provide a greater level of disaggregation than many large-N studies of rebellion, but also lacks information about the rebel groups beyond their network connections. I am not able to speak to the effects of the size, structure, or history of the group. Furthermore, the quantitative studies do not capture any information about the individuals participating in the rebellion, nor the connections between them beyond those that occur through group membership. One might imagine that the relationship between two dissident faction varies greatly depending on whether their members lack any intergroup connections, or whether many connections in some other network, perhaps coming from the same village. Case studies can also address the limited temporal domain of the quantitative analyses, illuminating, for example, whether external support had different effects during the Cold War.

One set of cases that is likely to provide useful evidence are the separatist movements of Burma. 11 different groups have attempted to secede since the country became independent in 1948. Comparing separatist movements in the same country has the advantage of holding several factors constant. These groups have similar goals (aside from the specific territory involved), they are fighting against the same government, they share similar colonial histories, and they have been active in roughly the same time periods. Yet, they show significant variation on the dependent variable, with some movements such as the Kokang and Karenni producing a single violent group, while the Arakan movement has been represented by seven violent groups, and the Shan by eleven. There is also significant variation on potential explanatory factors. Some of these groups are homogenous on many dimensions, while contain religious or linguistic divisions. Some groups have relationships with outside states or the country's communist insurgency, while others are isolated. There is variation in institutional structure as well, with some movements having robust political wings that engage in local governance, while other are comprised only of violent groups.

A potential technique for analyzing these cases is qualitative comparative analysis (QCA) \citep{Ragin2000}. QCA requires cases to be coded on the variables of interest, and identifies patterns in a manner similar to quantitative analysis. Where it departs from statistical approaches is in its focus on the precise form of the relationship between variables. For example, QCA is often used to identify necessary and sufficient conditions, or more complex causal chains. The Burmese separatist movements should provide an adequate number of cases for QCA, and the amount of information available on them should be sufficient to code the variables of interest. Alternatively, this analysis could take the form of a more open-ended qualitative case study. In this case, the focus would be on compensating for the weaknesses of the qualitative analysis, with an emphasis on tracing the processes through which rebel structures emerge and change.
