% !TEX root = 3_16.Rnw

\section{Dissident Network Structure and Patterns of Violence}

I argue that the structure of the dissident network should be a powerful determinant of the structure of rebel movements. Network ties should tend to capture the similarity of dissident interests, the crucial determinant of whether free rider incentives outweigh the value of mobilization for individual dissident groups. Additionally, in some cases networks indicate the existence of deeper social ties, which promote cohesion within rebel movements.

\subsection{Network Ties as a Measure of Preference Similarity}

A core assumption of this analysis is that the networked relationship between dissident groups captures the similarity (or dissimilarity) of their preferences \citep[a similar assumption is made by][]{Metternich2013}. This should be the case first because social networks in general tend to exhibit a high degree of homophily, meaning that the individuals or groups comprising the network nodes tend to form relationships on the basis of social similarity. For instance, individuals tend to form networks of friends with whom they share common race, age, religion, educational background, and ideological views \citep{Mcpherson2001}. Among dissident organizations, homophily is likely to take the form of similar ideological views, ethnicity, religion, and tactical preferences. I code network ties between dissidents whenever they are involved in verbal or material cooperation events,\footnote{This procedure is discussed in greater detail in the research design section below.} such as verbal endorsement of another group's goals, or collaboration on the battlefield. Cooperation of this sort should be likely to occur among groups with similar interests, and unlikely to occur among dissimilar groups. Furthermore, it is important to note that these ties are defined in this study entirely by the behavior of the dissidents themselves. Thus, network ties should be shaped primarily by the factors that the actors themselves consider to be important dividing lines in the conflict. In an ideological conflict, dissidents are likely to have close ties with groups holding similar beliefs, and weak or non-existent ties with groups with strongly differing beliefs. In conflicts where ascriptive identities play a central role, network structure is likely to reflect the key dividing lines. Thus, while I cannot make definitive claims about what precisely network ties represent, they should show a strong tendency to capture the most relevant social cleavages in a given conflict.

Networks offer an advantage over alternative approaches to measuring dissident in their ability to capture preference dynamics. Civil wars are often quite long, and change substantially in character over time. For instance, early in the Syrian Civil War dissident groups largely cooperated in their fight against the regime of Bashar al-Assad, with the relatively secular Free Syrian Army often collaborating with Jabhat at-Nusra, an al-Qaeda affiliate.\footnote{Weiss, Caleb. 2014. ``Free Syrian Army continues to cooperate with the Al Nusrah Front, Islamic Front in southern Syria.'' \url{http://www.longwarjournal.org/archives/2014/10/free_syrian_army_continues_to.php}} In later years of the conflict, the rise in prominence of the conflict between the Islamic State and US-backed groups appears to have driven a split between secular groups hoping to maintain support from the US, and Islamist groups.\footnote{Cambanis, Thanassis. 2016. ``The Syrian Revolution Against al Qaeda.'' \url{http://foreignpolicy.com/2016/03/29/the-syrian-revolution-against-al-qaeda-jabhat-al-nusra-fsa/}} The divide between secular and Islamist organizations did not appear to significantly shape dissident relations early in the conflict, while subsequently serving as a primary basis of conflict among dissidents. Similarly, the divide between moderates willing to accept a compromise with the government and hardliners may not be relevant to dissident relations early in a conflict, while driving fragmentation in the later stages when negotiated settlements begin to be discussed. Dissident relations may be endogenous to other conflict dynamics as well, such as the level and form of government repression \citep{McLauchlin2012}. These changes in dissident relations could not be easily captured by \textit{ex ante} measures of dissident group ideology or preferences, whereas they are easily accounted for in a dynamic network approach.

\subsection{Networks and Collective Action}

A successful rebellion produces a public good in the form of a new regime. This fact has led many scholars to speculate that participation in civil war is subject to a free rider problem at both the individual \citep{Lichbach1995,Wood2003,Bramoulle2007} and group \citep{Metternich2013} levels. It is true that replacing a despised regime with a more palatable one brings a non-excludable benefit to all who share such an opinion. For instance, Burundian dissidents opposed to President Nkurunziza's bid for an unconstitutional third term would benefit equally regardless of whether they personally participated in his ouster. Similarly, a resident of the Basque region would enjoy the benefits of independence from Spain regardless of whether they joined the ETA secessionist organization. Furthermore, one might expect the free rider problem to be especially acute in civil war, as participation in rebellion is an especially risky form of contribution to a public good. Rebels face the possibility of death or injury on the battlefield, and after the war if they are unsuccessful. Non-participation may avoid these risks, though \citet{Kalyvas2014} note that civilians also face significant risk during civil wars. \citet{Lichbach1995} concludes that collective action by dissidents is only likely if rebel groups offer private benefits such as revenues from natural resources or illicit activities to members. \citet{Wood2003}, by contrast, finds that strong ideological commitment was sufficient to motivate participation in the FMLN insurgency in El Salvador.

Yet, while the general occurrence of regime change is a non-excludable benefit for all dissidents, facets of the phenomenon create more private and rivalrous benefits. First, there is a distributional conflict over the precise character of the new regime \citep{Metternich2013}. The group(s) that are directly involved in the overthrow of an existing regime should tend to have disproportionate influence in shaping the new regime, though this may be contingent on the peace process. The new government will be most valuable to the former dissidents to whom it is closest it terms of ideology. For example, a communist insurgency would find much less benefit from free riding on a group hoping to establish a democratic-capitalist regime than it would leading a successful revolution of its own. Second, the individuals who hold positions of power are likely to receive private benefits \citep{Christia2012,Wolford}. Individuals may have opportunities to engage in corruption and or rent-seeking behavior, perhaps by pocketing a cut of oil revenues. Power itself is also likely to carry much value for the individuals who hold it, allowing them to shape policy. Individuals who played major roles in defeating the government should often have an advantage in obtaining these positions, though this is not guaranteed to be the case if elections occur quickly after the war.

These more particularistic benefits should work in opposition to the free rider problem, with free riding becoming less attractive as one's ideological and social distance from other dissidents increases. \citet{Bramoulle2014} model an individual's optimal contribution to a public good as being interdependent with the efforts of those in its social network. They find that the aggregate level of effort contributed is determined solely by the lowest eigenvalue of the network, which corresponds to the general level of social cohesion or fragmentation observed. Large lowest eigenvalues represent tight networks, which tend to have equilibria in which most or all actors contribute small amounts of effort, hoping to free ride off of their many closely-related fellow players. Small lowest eigenvalues correspond to sparse networks, and tend to have equilibria in which many players give maximal efforts as they do not benefit from the effort of other players. \citet{Metternich2013} find empirical support for these results, showing that the aggregate level of anti-government violence produced by Thai dissidents decreases as the dissident network becomes more tightly connected.

While \citet{Metternich2013} show that the structure of the dissident network affects the amount of violence it produces, they do not analyze the number of groups contributing to this violence. When preference diversity overpowers free rider incentives, dissidents should tend to form multiple rebel groups, rather than collaborate in a single, large group. Joining an existing group requires a dissident faction to navigate the internal politics of the group, winning influence for its priorities. Rebel groups tend to be either strongly hierarchical or lacking in institutions \citep{Staniland2014}, making them either unwilling or unable to credibly promise to accommodate a new faction's demands. Joining a group temporarily, and later breaking away to pursue separate interests may be a solution to this problem, though such a strategy does not guarantee a seat at the table in peace negotiations. The alternative of separate mobilization, meanwhile, is often relatively attractive. Groups that establish a significant presence on the battlefield stand a good chance of being included in the post-war peace process. Participants in such negotiations can often wield significant influence, as any sufficiently large group can act as a veto player \citep{Cunningham2006}. Although peace agreements that lack the involvement of some combatants are not uncommon, a rebel group that threatens to continue fighting can often exert significant leverage over those who wish to maximize the chances for peace. Even if a group is not able to secure its goals in peace negotiations, having already mobilized for violence, it retains the option of continuing to fight, including against other rebel groups.

It should be noted that some scholars make the opposite prediction regarding network connectivity and collective action \citep[e.g.][]{Marwell1988}. \citet{Kuran1991} expects dissidents to find strength in numbers, preferring to mobilize only when a sufficient number of like-minded individuals are also participating in collective action. \citet{Gould1993} addresses network structure more explicitly, arguing that individuals who are tightly networked with people who are already participating in collective action will feel pressure to join as well, as they wish to avoid violating norms of fairness. Yet, it is not clear that such mechanisms translate from the individual level to the group level. For instance, while individuals are indeed likely to find strength in numbers, it is not necessarily the case that a rebel faction will find its security significantly enhanced by partnering with another rebel group. Similarly, individual-level norms against free riding \citep{Gould1993} may not operate between groups.

In summary, I expect that dissidents will have an incentive to participate in conflict when their preferences differ substantially from those of existing violent groups. Given the frequently unaccommodating structure of many rebel groups, and the fact that decision rules in peace negotiations tend to grant considerable power to individual groups, dissidents should tend to form distinct rebel groups from dissidents with other preferences. When their preferences are similar to other dissidents, however, free riding is likely. As I argued at the outset of this section, the degree of preference similarity should be captured by the volume of network ties between dissident organizations. Networks with many ties should have similar interests, while networks with few ties should have more heterogeneous preferences.

\noindent \textit{H1: The number of rebel groups participating in a conflict should vary inversely with the density of ties in the broader dissident network.}

The effect of networks can vary beyond the general level of connectivity. Two networks with the same total number of ties could have very different structures. For instance, ties could be evenly distributed across a network, with each node being tied to one or two other nodes. Alternatively, the ties could be heavily concentrated in one region of the network, with a densely connected community of ties, and many other nodes with few or no connections to other ties. In the former case, each member of the dissident network has limited ability to free ride, as they are connected to one or two other groups. In the latter case, free rider incentives are heterogeneous, as members of the tight cluster have great ability to free ride due to their many links to other groups, whereas the groups that share no connections with other nodes are completely unaffected by the behavior of other groups. Thus, it is necessary to account for clustering in addition to network density. The current measure of clustering most preferred by network analysts is modularity \citep{Newman2004a}. This measure applies an algorithm to the network to sort nodes into communities, returning the maximum number of communities detected in the process.

Modularity is a separate dimension of network structure from density --- modular networks with low and high density of ties have different implications for free riding --- and thus its effect should be modeled as an interaction with, rather than an alternative to density. At all levels of tie density, greater modularity should be associated with higher numbers of rebel groups. Modular, low-density networks are likely to have many nodes with few or no connections to others. Modular, high-density networks are characterized by clusters with many internal ties, but few ties between clusters. In this case free riding should be rampant due to the volume of ties, but as the number of other groups that any dissident is connected remains relatively small, more groups should mobilize than in cases where ties are evenly distributed. Thus, the interaction of dissident network density and modularity should have a positive relationship with the number of rebel groups that mobilize.

\noindent \textit{H2: The number of rebel groups participating in a conflict should vary positively with the interaction of density and modularity in the broader dissident network.}

As assumption in the preceding discussion is that the possibility of participation in peace negotiations offers a strong incentive for dissident factions to participate in violence on an individual basis. This assumption is potentially testable, as it is often possible to identify periods in which negotiation is likely. The Integrated Crisis Early Warning System (ICEWS, described in greater detail in the research design section below) codes instances in which actors call on others to accept mediation, settle disputes, or de-escalate conflicts. If dissident factions mobilize independently to maximize their leverage in post-war bargaining, we might then expect the number of who are violently active to increase as calls to end the conflict increase.

\noindent \textit{H3: The number of rebel groups participating in a conflict should increase as calls for conflict resolution increase.}

\subsection{Networks and Social Context}

While it is widely accepted that network structures tend to reflect ideological similarity, not all network ties are created equal. Some reflect nothing more than fleeting alignments of convenience. Even if the dissidents in such a coalition have some degree of shared interest, they are generally unlikely to have a long history of interpersonal connection and the level of trust that often accompanies it, nor are they likely to have strong institutions to cement the relationship.
In other cases, however, deep social connections are embedded in networks. These might include family or clan ties, shared Church or Mosque membership, political party membership, or longstanding cooperation over agricultural work. When network structure reflects these sorts of ties, individual-level connections are likely, formal institutions often coordinate behavior, and informal institutions such as norms against shirking \citep{Gould1993} provide enforcement. Network ties with these attributes are likely to be both more stable and less likely to lead to free riding. Indeed, \citet{Staniland2014} finds that the structure of the social networks from which insurgent movements emerge is a powerful determinant of both movement structure and stability.

I have limited ability to distinguish between the horizontal ties between elites and vertical ties between elites and local communities discussed by \citet{Staniland2014}, nor do I have the ability to observe the ties within existing organizations. Still, cases at either extreme --- a high density of both horizontal and vertical ties, or an absence of both types of ties --- are likely to be reflected in the structure of the networks I observe. Thus it is possible to test the proposition that the initial structure of the dissident network is disproportionately influential in shaping rebel movements relative to the structures that form later in the conflict. Following Staniland's logic, rebel groups that emerge out of tightly networked social structures should exhibit greater capacity to withstand internal divisions, as well as greater ability to co-opt smaller dissident groups. Thus, networks that are dense prior to the start of a conflict should lead to conflicts with fewer rebel groups than conflicts that emerge out of looser dissident networks.

\noindent \textit{H4: The density of the dissident network at the beginning of a conflict should be inversely related to the number of rebel groups that participate in the conflict.}
