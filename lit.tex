\section{Theoretical and Empirical Contributions}

As noted at the outset, the primary contribution of this project is to advance our understanding of the causes of multi-dyadic civil wars, a particularly devastating subset of conflict. Though few, if any studies address this phenomenon in its entirety, several literatures explain specific processes that can lead to multiple rebel groups. I first situate this work within those literatures, before discussing the broader contributions.

\subsection{Fragmentation}

While no existing studies offer a complete explanation for the variation in the number of actors participating in conflict, several works address a subset of the issue, usually focusing on the fragmentation of previously coherent groups. One subset of fragmentation studies suggests that government behavior plays a key role in the unity of dissident movements. For instance, \citet{McLauchlin2012} find that government repression provides occasion for groups to evaluate their current leadership structure. Pre-existing divisions within groups are likely to be exacerbated, leading the group to move toward more factionalized leadership structures. When group members are satisfied, however, conflict tends to lead to even greater unity and centralization of authority. Similarly, \citet{Asal2012} find that groups with previously factionalized leadership structures are at much greater risk of splintering than groups with more consolidated governance. Whereas the preceding studies essentially treat government repression as exogenous to the internal politics of dissident groups, \citet{Bhavnani2011} present evidence that governments deliberately stoke tensions among their opponents, as they find that the Israeli government increased conflict between Fatah and Hamas by undermining Hamas' control of the Gaza and by tolerating Fatah's relationship with the Jordanian military.

Another group of scholars emphasizes concerns about post-conflict bargaining as the key determinant of dissident group cohesion. \citet{Christia2012} assumes that the winning coalition in a civil war receives private benefits, which might include any rents available to the state, or having some portion of its interests represented in the new government. Thus, rebels have an incentive to form coalitions that are at or only slightly above the minimum size needed to defeat the government, so as to maximize their share of the benefits. \citet{Wolford} develop a similar logic, theorizing that political factions have an interest in joining conflicts so as to maximize the likelihood of their preferences being represented in the post-war government, but the value of fighting decreases as the number of parties with whom they expect to share power increases. Yet, \citet[Ch. 2]{Christia2012} suggests that this incentive to minimize coalition size is moderated by the risk of being outside the winning coalition, as there is a strong possibility of new waves of violence between victorious rebels and rival rebel factions. She thus expects coalitions to change frequently in response to battlefield events, with factions bandwagoning with battle winners and shifting away from losing coalitions. \citet{Findley2012} similarly find fragmentation to be most common among groups that have recently lost battles. This implies that fragmentation is essentially a process of weak actors becoming weaker.

A final category of explanations places the source of rebel group cohesion in underlying social structure. \citet{Staniland2014} argues that insurgent organizations will be most stable when their central leadership is able to exercise both vertical control over its rank-and-file members, and horizontal control over its constituent groups. This is most likely to occur when insurgencies draw from existing organizations with extant social ties of this sort, which might include former anti-colonial movements or ethnic political parties. Organizations are likely to fragment when constituent groups have a high degree of autonomy or control over individual members is limited \citep[Ch 2-3]{Staniland2014}. \citet{Asal2012} emphasize similar factors, arguing that organizations with factionalized leadership structures are at risk of fragmentation, while groups with more consolidated power structures will tend to remain cohesive. Finally, \citet{Warren2015} suggest that group size plays an important role, as small groups are able to police themselves and resolve conflicts, whereas larger groups are more likely to experience infighting.

Each of these studies makes an important contribution to our understanding of conflict complexity, and sheds light on the broader interests and organizational challenges present in rebel movements. Yet, while the fragmentation of existing groups accounts for a substantial portion of multi-dyadic conflicts, other processes are at work in the majority of cases. Indeed, only 26.6\%\footnote{These figures are calculated using data on conflict participation from \citep{Pettersson2015a} and actor attributes from \citep{ucdpactor}. I code a conflict episode as a separate war if it occurs following at least two calendar years of inactivity. Secessionist movements are treated as separate conflicts from bids to overthrow the central government, and separate from each other if they concern different territories.} of the rebel groups that join ongoing civil wars splintered from an existing group, and only 9.7\% are agglomerations of existing groups. Thus, nearly two-thirds of the groups that join conflicts\footnote{The pattern is even more stark if one looks at conflict-years, as over 95\% of conflict years with multiple government-rebel dyads include at least one rebel group that is neither a splinter organization nor the source of one.} (only 20\% of multi-dyadic conflicts have multiple rebel groups from the outset) do not appear to be the product of existing combatants reconfiguring, but rather are the result of an entirely new group of combatants entering the fray. I propose an integrated approach that accounts for both the fragmentation of existing groups, and the entry of new groups to the conflict.

\subsection{Contagion}

Few, if any, studies directly consider the phenomenon of new rebel groups joining ongoing conflicts. The literature on contagion is perhaps most relevant. \citet{Gleditsch2007} finds that transnational ethnic groups and political and economic linkages between states can provide channels for civil war to spread across international boundaries. Other scholars find that secessionist \citep{Ayres2000} and ethnic \citep{Lane2016} conflict often spread through processes of contagion, with the rebellion of one group seemingly inspiring those in neighboring areas to take up arms themselves. Such transnational processes might shape opportunities for multiple rebellions to emerge by increasing the availability of weapons, spreading tactical knowledge, or diverting government attention to foreign conflicts. Similarly, transnational motives for conflict may come in the form of grievances becoming clearer and more salient in light of events in neighboring countries, as happened during the Arab Spring, or the expected probability of a successful rebellion shifting upward in response to nearby events. Yet, rebel groups that are themselves transnational, operating in multiple countries \citep[see][]{salehyan07} account for only 10.8\% of conflict joiners.

% Maybe conflict management?

In short, while the literatures on group fragmentation and contagion provide important insights to the preferences and organizational needs of rebel groups, these processes only partially overlap with the phenomenon of multi-dyadic conflict.

\subsection{Broader Contributions}

This project also contributes to a number of larger discussions in the civil war literature. Both the theoretical arguments and empirical results presented here shed new light on a number of ongoing questions.

\subsubsection{Collective Action in Civil War}

First, I provide further evidence that the organization of rebellion is beset by collective action problems, and that the magnitude of these problems are conditional on the degree to which dissidents are networked with each other. The general existence of free riding problems in civil war has been the subject of debate among civil war scholars. In a rich contribution to civil war theory, \citet{Lichbach1995} argues that the benefits of a successful rebellion are non-excludable, while the individuals who participate face substantial private risk. He concludes that successful rebellions are only likely to emerge when the individuals who participate are provided with private benefits, such as shares of revenues from natural resources or illicit activities. \citet{Wood2003} similarly frames collective action as improbable given the personal risk involved, but finds that support for the insurgency in El Salvador was driven not by private benefits, but rather ideological and emotional commitment to the cause. \citet{Kalyvas2014} doubt the existence of collective action problems altogether, noting that in reality free riding is generally not an option. Often, individuals in war zones can choose only between facing a high risk of violence as a civilian, or as a rebel.

To gain leverage on this debate, I adopt the increasingly common perspective that incentives to engage in collective action are conditional on an individual (or group's) relationships with others. The core assumption of this approach is that  many non-excludable goods can vary in character, meaning that some individuals will benefit from their existence more than others. In the case of civil war, this means that while overthrowing the government is a public good, the qualities of the new regime are chosen by those who participate in collective action, and may or may not be to the liking of those who free ride. This implies, then, that attractiveness of free riding relative to participating in collective action dependents on who else is participating. Actors should thus prefer to free ride when actors nearby in their social network (assuming networks capture the relevant aspects of interest) have mobilized, and have incentive to engage in collective action when similar actors have not done so \citep{Bramoulle2007}. \citet{Metternich2013} apply this perspective to civil war, finding that the level of contribution to public goods (violence against the government) increases as the network of anti-government groups becomes less tightly connected. I extend this literature to show that a dissident group's choice between non-violent and violent tactics is conditional on the closeness of its relationships with other dissident groups. This indicates that in contrast to \citet{Kalyvas2014}, free riding is possible in civil war, at least at the group level. Additionally, it suggests that ideological similarity can work against collective action, whereas \citet{Wood2003} finds it to be a key force in motivating it.

Evidence that a dissident group's decision to participate in violence in interdependent with the decisions of others also has consequences far beyond the collective action literature. Most theories of civil war concern the decision to resort to violence, yet few account for such interdependencies. For instance, a number of scholars model rebellion as emerging out of escalatory interactions between a government and protest movement, with the government's use of repression being the primary determinant of outcomes \citep[e.g.][]{Lichbach1987,Moore1998}. Yet, the willingness of protesters to escalate to more risky tactics is likely dependent on whether other groups have already resorted to violence. Similarly, the grievance model of civil war \citep{Collier2004} emphasizes poor economic performance as a motive for rebellion, yet these concerns are likely to be broadly shared and thus particularly susceptible to free riding. Conversely, an ethnic considering violent mobilization in response to horizontal inequalities \citep[see][]{gurr70,Cederman2010} would likely not be influenced by the strategies of other ethnic groups, as they are pursuing more particularistic benefits.

\subsubsection{Rebel Motives}

An examination of the relationships between dissident groups is also likely to offer a new perspective on rebel motives. For the last 15 years, the literature on civil war has largely been dominated by debates over whether rebellion is fundamentally political, or done in pursuit of private benefits. The former views civil war as an effort to resolve economic or political inequality \citep{gurr70,Wood2003,Cederman2010}, and has been labeled as the ``grievance'' hypothesis \citep{Collier2004}. The latter is composed primarily of studies emphasizing the ``greed'' hypothesis \citep{Collier2004}, which view rebellion as little more than large-scale criminal activity aimed at bringing profits to its members \citep{mueller00,Lujala2005,Ross2004e}. Others have emphasized non-material private benefits as motive for individual participation in rebellion, such as the ability to act on family disputes or romantic rivalries \citep{Kalyvas2006}.

This political-private motive debate has yet to be definitively resolved. A number of scholars have found greater support for the greed hypothesis than for grievance, with the presence of natural resources being a stronger predictor of civil war than economic or political grievances \citep{Collier2004}. Yet, these findings are not robust across different types of resources or even different measures of the same resource \citep{Dixon2009a}. Furthermore, several scholars have found that political factors such as hierarchical relationships between ethnic groups \citep{Cederman2010} and poor economic performance \citep{Miguel2004a} exert a strong influence on civil war onset. Other scholars eschew the dichotomy altogether, suggesting that while private benefits are useful to rebel recruiting efforts, this does not preclude the possibility that rebel elites ultimately have political motives \citep{Lichbach1995,Weinstein2007}. Similarly, \citet{Lujala2010} finds that natural resources are associated with longer conflicts, implying that at least a portion of resource revenues are devoted to fighting rather than private benefits.

One factor that has limited progress on these questions of motive is the fact that the competing theories have been tested almost exclusively on a single outcome --- a binary measure of the occurrence of civil war at the national level. Studying the relationships between dissident groups and how they vary is likely to provide insight to underlying rebel motives. For instance, if rebellion is fundamentally about maximizing the profits of its members, the structure of rebel movements should be shaped largely by the natural resources present in a country. As many of the extraction of many of the resources thought to be associated with rebellion is not particularly labor-intensive (for instance, the extraction of alluvial diamonds from river beds), rebels should prefer to keep their groups small so as to maximize the share of profits given to each individual member. In resource-rich areas, we should see highly fragmented rebel movements. Furthermore, the use of violence in such a scenario is likely to be aimed at acquiring and defending access to resources, and if anything is more likely if other groups have adopted violent tactics. If rebellion is a political enterprise, by contrast, there should be possibilities for coalition building and cooperation between rebels, as the primary concern is maximizing the likelihood of defeating the government. Yet, there is also a possibility for free riding, suggesting the decision to resort to violence is interdependent in the direction of becoming less likely as other groups mobilize.

\subsubsection{Civil Wars and International Context}

Finally, this project advances our understanding of the relationship between civil war and international politics. Evidence that localized international factors influence civil war is abundant. A country's risk of civil war is determined in part by the regime types of neighboring states, the presence of transnational ethnic group, and the occurrence of conflict nearby \citep{Gleditsch2007}. Many rebel groups operate transnationally, directly spreading the risk of violence \citep{salehyan07}. Deliberate actions by outside states also have powerful effects on the attributes of civil wars. Civil wars with third party intervention tend to last longer than those without \citep{Balch-Lindsay2000,Regan2002}, particularly when both sides receive support. Yet, the mechanisms by which outside intervention influences the conflict process are poorly understood. Prior studies assumed dyadic competition between a government and single rebel group, and suggested that capability enhancement was primarily responsible for the relationship. While that is certainly a key component of the story, this study suggests that external actors can influence the structure of dissidents as well. This has significant implications for conflict management, as it suggests that persuading outside states to end their involvement in a conflict will be insufficient for its resolution, as intervention may also increase the complexity of the underlying conflict.

The findings here also offer clues to the logic of intervention by outside states. While the determinants of intervention primarily aimed at pacifying conflict have received considerable attention \citep[e.g.][]{Regan2000a}, few studies examine the conditions under which more self-interested intervention occurs. The finding that external support tends to produce ``spillover'' effects whereby the network of dissidents within a conflict begins to resemble the network of supporters provides further support for the notion that support is often given through a proxy war logic \citep[see][]{Salehyan2010}. I then extend that line of thinking to examine the effects of external support on patterns of violence.
