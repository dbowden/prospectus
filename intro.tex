\section{Introduction}

Theories of civil war tend to focus on individual- or group-level motives \citep[e.g.][]{gurr70,Collier2004} or opportunities \citep[e.g.][]{fearonlaitin03} for rebellion, while giving little attention to the organization of dissent into rebel groups. Even those studies which do explicitly consider rebel group formation tend to focus on group attributes and do not consider the possibility that multiple groups might emerge \citep[e.g.][]{Weinstein2007}. Yet, this step in the conflict process is far from straightforward, as 44\% of civil conflicts feature at least two rebel groups challenging the government.\footnote{Source: \citet{Pettersson2015a}.} Over the course of the Chadian Civil War, for instance, 25 distinct rebel groups fought against the government. Conflicts in Afghanistan in the 1980's, Sudan in the 2000's, and Somalia in the 1990's have been similarly complex. Depending on the data source one uses, the ongoing civil war in Syria is contested by at least two dozen, and perhaps as many as 200 armed groups. Even ethnically homogeneous, geographically concentrated movements with common goals, such as the Karen secessionist campaign in Myanmar, often fragment into multiple rebel groups. The existing literature offers many useful insights to the conditions under which civil war will emerge, but it has few explanations of the organization of dissent into specific rebel groups.

While little attention has been given to the causes of multi-dyadic conflict, several studies suggest that such configurations can have deleterious consequences. Conflicts with multiple rebel groups last longer than dyadic competitions \citep{Cunningham2006,Cunningham2009,Akcinaroglu2012a}. \citet{Cunningham2009} find that the presence of multiple government-rebel dyads decreases the likelihood of peace agreements and increases the likelihood of rebel victories, though \citet{Findley2012}, find that fragmented rebel movements are often associated with an \textit{increased} likelihood of negotiated settlement. Relatedly, \citet{Atlas1999} find that episodes of conflict renewal often occur between formerly allied rebel factions. Finally, conflicts with multiple dyads feature more fatalities than dyadic ones.\footnote{Source: my own analysis using data from \citep{Sundberg2008a}.} Clearly, conflicts with multiple rebel groups comprise one of the most severe subsets of civil wars. Thus, understanding the causes of multi-dyadic conflict is of great normative and policy importance.

I seek to resolve this gap in the literature and enhance our understanding of multi-dyadic civil war by addressing two broad questions. First, why do some conflicts feature a multitude of rebel groups, while the majority are contested by just one? Second, how do the factors that shape rebel movements change over time? I argue that understanding the emergence of multiple rebel groups requires consideration of the broader network of dissidents in a conflict. A dissident group's relationships with others are a powerful determinant of its incentives to mobilize individually, or support the efforts of others.

This work extends and addresses a number of existing literatures. In recent years studies of conflict have employed analyses disaggregated on several dimensions, including by actor \citep[e.g.][]{Cunningham2009,Pearlman2011a,Fjelde2012}. Whereas the existing literature is primarily concerned with the fragmentation of existing actors, this study expands our understanding of multi-dyadic conflict by considering the entry of entirely new actors. Second, examining the relationships between rebel groups sheds new light on debates about the true motives behind rebellion \citep[e.g.][]{Collier2004}. Finally, it extends the literature on civil war intervention to demonstrate new area in which outside actors can influence conflict --- the structure of the warring parties.

I proceed with a more detailed discussion of this project's contributions to the existing literature. Subsequently, I articulate a theoretical framework that establishes the set of dissidents from which rebel groups emerge. Next I offer a theory relating the structure of this dissident network to the number of rebel groups that participate in a conflict, followed by a theory of what shapes the dissident network itself. After detailing a research design for testing these propositions, I provide two pilot studies. The first demonstrates the dissident network approach in the context of the Israel-Palestine conflict. The second uses a panel analysis to identify a link between international support, a factor that I argue is an important determinant of the evolution of the dissident network, and the formation of new rebel groups.
